\batchmode
\documentclass[12pt]{article}
\usepackage{ngerman}
\usepackage[latin1]{inputenc}
\usepackage[pdftex]{graphicx}
\parindent0pt

\begin{document}

%\newpage
%\tableofcontents
%\newpage

\section{Datenstrukturen}
\subsection{Grid}

\subsubsection{obstacles}
Bool'sche Matrix mit true f�r ein obstacle und false sonst.

\subsubsection{h}
H�henfeld.

\subsubsection{h\_index}
Ent"alt zu jeder Stelle in h den korrespondierenden Index im PackedGridData h Vektor.

\subsection{PackedGridInfo}

\subsubsection{offset}
Bezeichnet die Stelle des ersten Elements der data Vektoren im globalen data Vektor.

Beispiel: lokal\_data[0] entspricht global\_data[offset].

\subsubsection{limits}
Enth�lt die kontinuierlich elementweise abzuarbeitenden Elemente der data Vektoren (limits[0] bis limits[size - 1]).
Hierbei geht ein Abschnitt von limits[i] bis eins vor limits[i + 1].
Der n�chste Abschnitt beginnt mit limits[i+1].
Die data Vektoren k�nnen auch vor limits[0] und nach limits[size - 1] Elemente haben, wenn PackedGridInfo aus dem GridPartitioner stammt.

\subsubsection{types}
Gibt f�r jeden Eintrag in limits den Typ der in diesem Abschnitt liegenden Elemente an.
Das erste Bit kodiert eine Boundary in Direction 1 usw.

\subsubsection{cuda\_types}
Gibt f"ur jedes Element der data Vektoren den Typ an (ist nur bei einem tags::GPU::CUDA L"oser aktiviert).

\subsubsection{dir\_index\_n}
Enth�lt die Abschnitte, in denen Zelle i einen Nachbarn in Richtung n besitzt.
Hierbei geht der Bereich von i = dir\_index\_n[x] bis i $<$ dir\_index\_n[x + 1] exklusive.
Der n�chste Bereich beginnt dann bei dir\_index\_n[x + 2].

\subsubsection{dir\_n}
Gibt f�r den Bereich der bei i = dir\_index\_n[x] beginnt die in dieser Richtung liegende Zelle j = dir\_n[x/2] an.

Beachte: dir\_n.size() = 0.5 * dir\_index\_n.size().

\subsubsection{cuda\_dir\_n}
Gibt f"ur jedes Element der data Vektoren die benachbarten Zellen an an (ist nur bei einem tags::GPU::CUDA L"oser aktiviert).

\subsection{PackedGridData}

\subsection{PackedGridFringe}

\subsubsection{h\_index}
Enth�lt die Abschnitte, in denen h, u und v von anderen L�sern beschrieben werden.
Hierbei geht der Bereich von i = h\_index[x] bis i $<$ h\_index[x + 1] exklusive.
Der n�chste Bereich beginnt dann bei h\_index[x + 2].

\subsubsection{h\_targets}
Enth�lt den L�ser, der den gew�nschte Bereich von u, h und v  beschrieben hat.

Beachte: h\_targets.size() = 0.5 * h\_index.size().

\subsubsection{dir\_index\_n}
Enth�lt die Abschnitte, in denen der f\_temp\_n Vektor anderen L�sern mitgeteilt werden muss.
Hierbei geht der Bereich von i = dir\_index\_n[x] bis i $<$ dir\_index\_n[x + 1] exklusive.
Der n�chste Bereich beginnt dann bei dir\_index\_n[x + 2].

\subsubsection{dir\_targets\_n}
Enth�lt den L�ser, dem der Bereich von f\_temp\_n mitgeteilt werden muss.

Beachte: dir\_targets\_n.size() = 0.5 * dir\_index\_n.size().

\subsubsection{external\_dir\_index\_n}
Enth�lt die Abschnitte, in denen der f\_temp\_n Vektor von anderen L"osern beschrieben wurde.
Hierbei geht der Bereich von i = dir\_index\_n[x] bis i $<$ dir\_index\_n[x + 1] exklusive.
Der n�chste Bereich beginnt dann bei dir\_index\_n[x + 2].

\section{Grid Tools}
\subsection{GridPacker}
\subsubsection{pack}
Erzeugt aus einem Grid Objekt PackedGridData und PackedGridInfo.

\subsubsection{unpack}
Schreibt den data.h Vektor wieder in das Grid H"ohenfeld zur"uck.

\subsubsection{h\_index}
Gibt zu einer Stelle i, j in einem Grid H"ohenfeld den Index des data.h Vektors zur"uck.

\subsection{GridPartitioner}
\subsubsection{compose}
Schreibt die H"ohenfelder aller Patches wieder in das urspr"ungliche data.h Feld zur"uck.

\subsubsection{decompose}
Zerlegt PackedGridData und PackedGridInfo in mehrere unabh"angige Patches.

\subsubsection{synch}
Synchronisiert die f\_temp Vektoren und den h Vektor zwischen allen Patches.

\section{L�ser}

\end{document}
