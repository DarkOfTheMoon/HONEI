\documentclass[a4paper,DIV12]{scrartcl}

\usepackage[latin1]{inputenc}

\usepackage[american]{babel}
\usepackage[dvips]{graphicx}

\usepackage{ifpdf}


\setlength{\parindent}{1em}

\ifpdf
  \usepackage[pdftex,colorlinks=true,urlcolor=blue,linkcolor=blue]{hyperref}
  \pdfcompresslevel=9
\else
%  \usepackage{t1enc}
  \usepackage{url}
%  \usepackage[dvips]{hyperref}
\fi


\title{SPAI -- SParse Approximate Inverse Preconditioner}
\author{Marcus Grote
  \thanks{Department of Mathematics, University of Basel,
    Rheinsprung 21, 4051 Basel, Switzerland, {\tt Marcus.Grote @unibas.ch}.}
\and Michael Hagemann
  \thanks{Department of Computer Science, University of Basel.}}

% \date{2005-09-30}




\begin{document}


\maketitle
\tableofcontents


\section{Introduction}

Given a sparse matrix $A$ the SPAI Algorithm computes a sparse approximate
inverse $M$ by minimizing $ \| AM - I \| $ in the Frobenius norm. The
approximate inverse is computed explicitly and can then be applied as a
preconditioner to an iterative method. The sparsity pattern of the
approximate inverse is either fixed a priori or captured automatically:
\begin{description}

  \item[Fixed sparsity] The sparsity pattern of $M$ is either banded or
    a subset of the sparsity pattern of $A$.

  \item[Adaptive sparsity] The algorithm proceeds until the 2-norm of each
    column of $AM-I$ is less than \emph{eps}.  By varying \emph{eps} the
    user controls the quality and the cost of computing the preconditioner.
    Usually the optimal \emph{eps} lies between 0.5 and 0.7.

\end{description}
A very sparse preconditioner is very cheap to compute but may not lead to
much improvement, while if $M$ becomes rather dense it becomes too expensive
to compute.  The optimal preconditioner lies between these two extremes and
is problem and computer architecture dependent.

The approximate inverse $M$ can also be used as a robust (parallel) smoother
for (algebraic) multi-grid methods.  For further details please refer to
\cite{7,8,9}.


\section{News}

Please see the \texttt{NEWS} file for a detailed list of changes.  Some
notable improvements over version 3.1 include:
\begin{itemize}
  \item Simplified installation procedure (\texttt{autoconf} based)
  \item Threshold based construction of preconditioner, with new
    \textsc{Matlab} interface function \texttt{spaitau}.
  \item Banded structure of preconditioner, with new \textsc{Matlab}
    interface function \texttt{spaidiags}.
\end{itemize}


\section{Installation}

\subsection{For the impatient}

The following gives you the basic commands for a quick build of SPAI, and
illustrates some simple ways to get you started with using SPAI as a
preconditioner:

\begin{description}
  \item[Step 1:]
    % Decide where your \texttt{installdir} will be. e.g. for a local
    % installation you may decide to use
    % \texttt{\$$\{$HOME$\}$/spaiinstalldir}

    Unpack the SPAI package, and change to the respective directory:
    \begin{quote}
\begin{verbatim}
gunzip spai-3.2.tar.gz
tar xvf spai-3.2.tar
cd spai-3.2
\end{verbatim}
    \end{quote}

  \item[Step 2:]
    Run the following sequence and keep your fingers crossed:
    \begin{quote}
        % --prefix=${HOME}/spaiinstalldir
        % make install
\begin{verbatim}
configure --with-matlab
make
make check
\end{verbatim}
    \end{quote}

    (Please note that a working \textsc{Fortran} compiler is required to
    detect the system calling conventions.
    %
    Furthermore, \textsc{Matlab} needs to be in the current \texttt{\$PATH}
    in order to be detected and used.  See Section \ref{sec:platforms} for
    further configuration options on various platforms.)

  \item[Step 3:]
    % Look at \texttt{ls -R (installdir)}

    Test the command line interface:
    \begin{quote}
\begin{verbatim}
cd src
./spai ../data/m1.mm
./spai ../data/m1.mm -sc 1 -ta 0.7
\end{verbatim}
    \end{quote}
    (Please note that the matrix file needs to be the first parameter.)

  \item[Step 4:]
    % Have fun using SPAI!

    Test the \textsc{Matlab} interface (if available):
    \begin{quote}
% >> % A is simple Poisson problem, b=A*1
% >> A = gallery('poisson', 100);
\begin{verbatim}
cd ../matlab
matlab
>> help spai
>> % Load A as matrix `m1.dat' and create RHS b=A*1.
>> A = spconvert(load('m1.dat'));
>> b = A * ones(size(A,1), 1);

>> % Solve Ax=b, without preconditioning
>> x = bicgstab(A, b, 1e-8, 200);

>> % Compute adaptive SPAI preconditioner (SPAI 3.1)
>> M = spai(A);
>> spy(M);

>> % Solve Ax=b, with SPAI preconditioning
>> x = bicgstab(A, b, 1e-8, 200, @(y) M*y);

>> % Compute block SPAI preconditioner (block size=3)
>> M = spai(A, 0.6, 5, 5, 3, 100000, 0);

>> % Compute threshold based SPAI preconditioner
>> M = spaitau(A, 0.8, 0);

>> % Compute banded SPAI preconditioner
>> M = spaidiags(A, 5, 5, 0);
\end{verbatim}
    \end{quote}

\end{description}

See Figure \ref{fig:matlab-spy} for a comparison of the sparsity patterns
for the above Matlab function calls, and the convergence history of a
preconditioned BICGSTAB iteration.

\begin{figure}[p]
    \centering
    \includegraphics[width=0.9\linewidth]{fig-m1}
    \caption{Comparison of sparsity patterns for different SPAI Matlab
      functions, and convergence history of preconditioned BICGSTAB
      iteration.}
    \label{fig:matlab-spy}
\end{figure}

% Note that this will install a copy of the libraries, the binaries, the
% mexfiles and the documentation in the subdirectory \texttt{spaiinstalldir}
% of your home directory. It also compiles versions of BLAS and LAPACK that
% come with the SPAI-package.

% This is a good start for experimenting with SPAI. Nonetheless for large
% matrices this building procedure may be improved by using vendor supplied
% BLAS and LAPACK libraries and using optimal compiler flags.

\subsection{Detailed installation instructions}

% \subsubsection{Directory tree of package}

% This package consists of the following subdirectories:

% \begin{description}
%   \item[\texttt{src}]
%     Contains code for SPAI library. This can be compiled as either a
%     parallel MPI version or a serial version that doesn't require MPI
%     (default).

%     Also contains a main program for testing SPAI. The executable "spai"
%     will read a matrix $A$ and (optionally) a right hand side $b$, construct
%     the SPAI preconditioner $M$ using a combination of various options, and
%     run BICGSTAB to solver $Ax=b$.

%   \item[\texttt{matlab}]
%     Contains an interface to Matlab (supports only the serial version of
%     SPAI).

%   \item[\texttt{data}]
%     An example matrix and rhs.

%   \item[\texttt{doc}]
%     Contains this documentation.

% \end{description}


\subsubsection{Using configure}

The `\texttt{configure}' shell script attempts to guess correct values for
various system-dependent variables used during compilation.  It uses those
values to create a \texttt{Makefile} in each directory of the package.  It
may also create one or more \texttt{.h} files containing system-dependent
definitions.  Finally, it creates a shell script \texttt{config.status} that
you can run in the future to recreate the current configuration, a file
\texttt{config.cache} that saves the results of its tests to speed up
reconfiguring, and a file \texttt{config.log} containing compiler output
(useful mainly for debugging \texttt{configure}).

% If you need to do unusual things to compile the package, please try to
% figure out how \texttt{configure} could check whether to do them, and
% mail diffs or instructions to \texttt{broeker@inf.ethz.ch}.

% If at some point \texttt{config.cache} contains results you don't want to
% keep, you may remove or edit it.

% The file \texttt{configure.in} is used to create \texttt{configure} by a
% program called \texttt{autoconf}.  You only need \texttt{configure.in} if
% you want to change it or regenerate \texttt{configure} using a newer
% version of \texttt{autoconf}.

The simplest way to compile this package is:

\begin{enumerate}
  \item \texttt{cd} to the directory containing the package's source code
    and type \texttt{./configure} to configure the package for your
    system.  If you're using \texttt{csh} on an old version of System V,
    you might need to type \texttt{sh ./configure} instead to prevent
    \texttt{csh} from trying to execute \texttt{configure} itself.

    Running \texttt{configure} takes a while.  While running, it prints
    some messages telling which features it is checking for.

  \item Type \texttt{make} to compile the package.

  \item Optionally, type \texttt{make check} to run any self-tests that
    come with the package.

  \item Type \texttt{make install} to install the programs and any data
    files and documentation.

  \item You can remove the program binaries and object files from the source
    code directory by typing \texttt{make clean}.  To also remove the
    files that \texttt{configure} created (so you can compile the package
    for a different kind of computer), type \texttt{make distclean}.

    % There is also a \texttt{make maintainer-clean} target, but that is
    % intended mainly for the package's developers.  If you use it, you may
    % have to get all sorts of other programs in order to regenerate files
    % that came with the distribution.
\end{enumerate}


\subsubsection{Optional Features}

\texttt{configure} accepts the following command line options:

\begin{description}
  \item[\texttt{-{}-help}]
    Print a summary of the options of \texttt{configure}, and exit.


  \item[\texttt{-{}-with-blas=\{no|"-llibrary1 -llibrary2"\}}]

    Override the default option for BLAS libraries.  By default, various
    known and vendor specific BLAS libraries are searched for and tested.
    This search can be overridden by specifying the BLAS linker options.

    \textbf{Example:} To build with a free \textsc{Atlas} BLAS installed in
    a non-standard place, try the following:
\begin{verbatim}
   ./configure --with-blas="-L/opt/atlas -lf77blas -lcblas -latlas"
\end{verbatim}

    If no suitable system BLAS library is found, a local library is built.
    This will typically decrease the performance of \texttt{spai}.


  \item[\texttt{-{}-with-lapack=\{no|"-llibrary1 -llibrary2"|/path/to/lapack-libary.a\}}]

    Override the default option for LAPACK libraries.  By default, various
    known and vendor specific LAPACK libraries are searched for and tested.
    This search can be overridden by specifying the LAPACK linker options.

    If no suitable system LAPACK library is found, a local library is built.
    This will typically decrease the performance of \texttt{spai}.


  \item[\texttt{-{}-with-matlab}]

    Checks for a \textsc{Matlab} installation and compiles the
    \textsc{Matlab} interface.  \textsc{Matlab} must be in the current
    \texttt{\$PATH}.


  \item[\texttt{-{}-with-MPI=\{lam|mpich|generic\}}]
    Compiles a parallel version using an already installed version of the
    MPI library.

    The value \texttt{lam} configures the package using specific details
    of the LAM implementation of MPI. See the {\em LAM homepage}
    \url{http://www.lam-mpi.org/} for details on how to obtain, install and
    run LAM.

    The value \texttt{mpich} configures the package using specific
    details of the MPICH implementation of MPI. See the {\em MPICH homepage}
    \url{http://www-unix.mcs.anl.gov/mpi/mpich/} for details on how to
    obtain, install and run MPICH.

    The value \texttt{generic} tries to guess the needed libraries and
    binaries needed.

\end{description}



\subsubsection{Compilers and Options}
\label{sec:platforms}

Some systems require unusual options for compilation or linking that the
\texttt{configure} script does not know about.  You can give
\texttt{configure} initial values for variables by setting them in the
environment.  Using a Bourne-compatible shell, you can do that on the
command line like this:
\begin{quote}
\begin{verbatim}
CC=c89 CFLAGS=-O2 LIBS=-lposix  ./configure
\end{verbatim}
\end{quote}

Or on systems that have the \texttt{env} program, you can do it like this:
\begin{quote}
\begin{verbatim}
env CC=c89 CFLAGS=-O2 LIBS=-lposix  ./configure
\end{verbatim}
\end{quote}
%
We have successfully installed and tested SPAI on the following systems:

\begin{itemize}

  \item  \textbf{i686-pc-linux-gnu} with GNU \texttt{gcc}

    All features are typically discovered automatically.  A working
    \textsc{Fortran} compiler needs to be installed.
    %
    Tested for \texttt{libspai.a} library, CLI, Matlab-interface, and MPI
    (LAM).

  \item  \textbf{sparc-sun-solaris2.8} with SUN \texttt{cc} and SUN high performance library

    We used the following options to avoid linking problems due to the
    \textsc{OpenMP} stubs.
\begin{verbatim}
  CC=cc F77=f90 CFLAGS="-xopenmp=stubs"  ./configure
\end{verbatim}
    The \texttt{sunperf} libraries are detected automatically.  If you want
    to compile the \textsc{Matlab} interface, you cannot link to the
    \texttt{sunperf} libraries.  A possible solution is to compile the local
    BLAS and LAPACK libraries with:
\begin{verbatim}
  CC=cc F77=f90 CFLAGS="-xopenmp=stubs"  ./configure  \
        --with-matlab --with-blas=no --with-lapack=no
\end{verbatim}

  \item  \textbf{x86\_64-unknown-linux-gnu}

    We use the \textsc{AMD} \texttt{ACML} library:
\begin{verbatim}
  ./configure --with-blas=/opt/acml/gnu64/lib/libacml.a
\end{verbatim}

    With \textsc{Matlab}, the following options were necessary, in order to
    make the library dynamically linkable:
\begin{verbatim}
  CFLAGS=-fpic ./configure --with-matlab  \
                           --with-blas=/opt/acml/gnu64/lib/libacml.a
\end{verbatim}

  \item  \textbf{powerpc-apple-darwin8.2.0} (with gcc)

    All features are typically discovered automatically.
    %
    Tested for library, and CLI.

\end{itemize}



\subsection{Installation}

Automatic installation is available through the \texttt{automake} package
(\texttt{make install}), but we recommend a local installation.  If the
library needs to be accessible to multiple users, the \texttt{libspai.a}
file can be copied to an appropriate location (\texttt{/usr/local/lib}).


\subsubsection{Matlab interface}

If you want to use the \textsc{Matlab} interface outside of the
\texttt{spai-3.2/matlab} directory, the simplest way is to include the
directory in the \texttt{\$MATLABPATH} environment variable.

Inside \textsc{Matlab}, this is possible with the \texttt{addpath} and
\texttt{path} functions:
\begin{itemize}

  \item Prepend to \texttt{MATLABPATH}:
\begin{verbatim}
  >> addpath /dir1 /dir2
\end{verbatim}

  \item Append to \texttt{MATLABPATH}:
\begin{verbatim}
  >> p = path()
  >> path (p, '/dir1')
\end{verbatim}

\end{itemize}

You can include these commands in the \texttt{~/matlab/startup.m} file,
which is automatically loaded every time \textsc{Matlab} is started.
%
You can also change the path permanently by setting the environment variable
\texttt{\$MATLABPATH} in your shell startup script.


% By default, \texttt{make install} will install the package's files in
% \texttt{/usr/local/bin}, \texttt{/usr/local/man}, etc.  You can
% specify an installation prefix other than \texttt{/usr/local} by giving
% \texttt{configure} the option \texttt{-{}-prefix=PATH}.

% You can specify separate installation prefixes for architecture-specific
% files and architecture-independent files.  If you give \texttt{configure}
% the option \texttt{--exec-prefix=PATH}, the package will use PATH as the
% prefix for installing programs and libraries.  Documentation and other data
% files will still use the regular prefix.

% In addition, if you use an unusual directory layout you can give options
% like \texttt{--bindir=PATH} to specify different values for particular
% kinds of files.  Run \texttt{configure --help} for a list of the
% directories you can set and what kinds of files go in them.

% If the package supports it, you can cause programs to be installed with an
% extra prefix or suffix on their names by giving \texttt{configure} the
% option \texttt{--program-prefix=PREFIX} or \texttt{--program-suffix=SUFFIX}.


\subsubsection{Installed files}

If you use the \texttt{make install} command for the installation, the
following files are installed in the \texttt{\$PREFIX} directory (the
default is \texttt{/usr/local/}):
\begin{verbatim}
  $(PREFIX)/bin
  $(PREFIX)/bin/spai
  $(PREFIX)/bin/convert
  $(PREFIX)/lib
  $(PREFIX)/lib/libspai.a
  $(PREFIX)/matlab
  $(PREFIX)/matlab/spai
  $(PREFIX)/matlab/spai/spai_full.mexglx
  $(PREFIX)/matlab/spai/spai.m
  $(PREFIX)/matlab/spai/spaitau.m
  $(PREFIX)/matlab/spai/spaidiags.m
  $(PREFIX)/share
  $(PREFIX)/share/doc
  $(PREFIX)/share/doc/spai
  $(PREFIX)/share/doc/spai/spaidoc.ps
  $(PREFIX)/share/doc/spai/spaidoc.pdf
\end{verbatim}
You can change the \texttt{\$PREFIX} directory with the \texttt{-{}-prefix=}
option of the \texttt{configure} script.


\subsection{Usage}

\subsubsection{Test program "spai"}

The test program is called "spai" and \texttt{make install} will install it
in your \texttt{\$prefix/bin} directory. You can set your \texttt{PATH}
environment variable accordingly.

The program will
\begin{enumerate}
  \item  Read a sparse matrix $A$
  \item  Optionally read a rhs vector $b$ (or construct one)
  \item  Construct a SPAI preconditioner $M$
  \item  use BICGSTAB to solve $Ax = b$, using $M$ as a preconditioner.
\end{enumerate}

The serial program is executed with:
\begin{quote}
\begin{verbatim}
spai <A> [b] [options]
\end{verbatim}
\end{quote}

while the parallel version is executed with:
\begin{quote}
\begin{verbatim}
mpirun -np <n> ./spai <A> [b] [options]
\end{verbatim}
\end{quote}

(Items in \texttt{$<$$>$} are required; items in {[}] are optional.)


\texttt{$<$A$>$} must be a file in Matrix Market coordinate format. See
the file data/m1.mm for an example. (Matrix Market is a repository of sparse
matrices on the web.  The only Matrix Market format currently supported is
"real general". This format is very similar to the MATLAB sparse matrix
format.  See the {\em MatrixMarket Homepage}
\url{http://math.nist.gov/MatrixMarket/} for details.

\texttt{[b]} is an optional dense vector giving the right-hand-side for the
BICGSTAB solver. It must be a file in Matrix Market array format. See the
file \texttt{data/m1\_rhs.mm} for an example. If the RHS is not given then
BICGSTAB will will use $A * 1\!\!1$ as a RHS, where $1\!\!1$ is a vector of
all ones.  The solution of the system $Ax=b$ will be written to the file
\texttt{solution.mm}.

\medskip

There are a number of optional parameters.  They all have default values and
are of the form \texttt{-xx s}, where \texttt{xx} is a two-letter
abbreviation of the parameter's name, and \texttt{s} determines its value.
Usually only a few need to be modified in practice.

\begin{description}

  \item[-sc]  sparsity control;  default is 0

    0 = adaptive \{\texttt{-ep}, \texttt{-mn}, \texttt{-ns}, \texttt{-bs} \} \\
    1 = specified tau \{ \texttt{-ta} \},\\
    2 = fixed diagonals \{ \texttt{-ud}, \texttt{-ld} \}

    These options for sparsity control are mutually exclusive.\\
    The main diagonal is always included.

  \item[-ep]
    $eps$ parameter for SPAI;  default is 0.6

    $eps$ must be between 0 and 1.  It controls the quality of the
    approximation of $M$ to the inverse of $A$.  Higher values of $eps$ lead
    to more work, more fill-in, and usually better preconditioners.

    % In many cases the best choice for $eps$ is the one that divides the
    % total solution time equally between the preconditioner and the solver.

  \item[-bs]
    block size;  default is 1

    A block size of 1 treats $A$ as a matrix of scalar elements. A block
    size of $s > 1$ treats both $A$ and $M$ as $s \times s$ block
    matrices.  A block size of 0 treats A as a matrix with variable sized
    blocks, which are determined by searching for dense square diagonal
    blocks in A.  This can be very effective for finite-element matrices.

    SPAI will convert A to block form, use a block version of the
    preconditioner algorithm, and then convert the result back to scalar
    form.

    In many cases a block-size parameter $s \neq 1$ can lead to significant
    improvements in performance.

  \item[-ns]
    maximum number of improvement steps per row in SPAI;  default is 5

    SPAI constructs an approximation to every column of $A^{-1}$ in a series
    of improvement steps.  The quality of the approximation is determined by
    $eps$.  If a residual less than $eps$ is not achieved after \texttt{ns}
    steps, SPAI simply uses the best approximation obtained so far.

  \item[-mf]
    message file for warning messages;  default is 0 (\texttt{/dev/null})

    Suppose you are using the option \texttt{"-mf foo"}.  Whenever SPAI fails
    to achieve an approximation of epsilon for a column of M it writes a
    message to the file "fooi" where i is the MPI rank of the processor
    generating the message (or i=0 in the serial case).

  \item[-mn]
    maximum number of new nonzero candidates per step;  default is 5

  \item[-sp]
    symmetric pattern;  default is 0

    If A has a symmetric nonzero pattern use \texttt{-sp} 1 to improve
    performance by eliminating some communication in the parallel version.

  \item[-mb]
    size of various working buffers in SPAI;  default is 100000

    Increase this if you run into problems with crashes.  100000 is a very
    generous size for most problems.

  \item[-cs]
    cache size $\{$0,1,2,3,4,5$\}$ in SPAI;  default is 5 (the biggest cache
    size)

    SPAI uses a hash table to cache messages and avoid redundant
    communication.  We recommend always using \texttt{-cs 5}.  This
    parameter is irrelevant in the serial version.

  \item[-mi]
    maximum number of iterations in BICGSTAB;  default is 500

  \item[-to]
    tolerance in BICGSTAB default;  is 1.0e-8

  \item[-lp]
    left preconditioning;  default is $0 \rightarrow$ right preconditioning

    SPAI stores matrices in a compressed-column format and by default
    computes a right preconditioner.  If the -lp 1 option is given SPAI will
    use a compressed-row format and compute a left preconditioner.

  \item[-bi]
    read matrix as a binary file;  default is 0

    There is a program called "convert" in the driver directory that
    converts an ASCII matrix file (in Matrix Market format) to a binary
    file. This can greatly speed up the input of large matrices.

    Try:
\begin{verbatim}
  convert  ../data/m1.mm  m1.binary
  spai  m1.binary  -bi 1
\end{verbatim}

  \item[-vb]
    verbose;  default is 1

    Print parameters, timings and matrix statistics.

  \item[-db]  debugging level;  default is 0

  \item[-ld]  \# of lower (below main) diagonals;  default is 0

    The main diagonal is always included.

  \item[-ud]  \# of upper (above main) diagonals;  default is 0


  \item[-ta]  (tau) Threshold for non-zero entries in $M$;  default is 0

    Only those entries $M_{ij}$ are included for which
    \[ |A_{ij}| > (1-tau) * max_{j} |A_{ij}| \textrm{, \ \ i.e.} \]

    $tau = 0$: main diagonal only,\\
    $tau = 1$: full pattern of A.

    The main diagonal is always included.
\end{description}



% \subsubsection{Matlab interface}

% Installation of the MATLAB interface may require some manual configuration
% of the compilation parameters of the mex compiler. The mex compiler is set
% up with \texttt{mex -setup}. Interactively configure \texttt{mex} to
% your needs. This will create a \texttt{mexopts.sh} file - most likely in
% a subdirectory of \texttt{\~{}/.matlab/}. Here is part of the output when
% we configured mex to use the \texttt{gcc}.

% \begin{quote}
% \begin{verbatim}
% > mex -setup
% [...]
% The options files available for mex are:

%   1: /pub/matlab/6.0_r12/bin/gccopts.sh :
%       Template Options file for building gcc MEX-files

%   2: /pub/matlab/6.0_r12/bin/mexopts.sh :
%       Template Options file for building MEX-files via the
%       system ANSI compiler


% Enter the number of the options file to use as your default options file:

% 1

% /home/broeker/.matlab/R12 does not exist. It will be created.


% /pub/matlab/6.0_r12/bin/gccopts.sh is being copied to
% /home/broeker/.matlab/R12/mexopts.sh
% \end{verbatim}
% \end{quote}

% Edit this file if necessary. Use \texttt{mex -v} to check that your local
% mexfile is really being and what parameters are being used in what order.

% Run \texttt{make all}. The matlab interface should compile.


\subsubsection{Running \texttt{make} in general}

\paragraph{Running checks}

You can run \texttt{make check} to perform some automatic checks to see
if the libraries and the optional MATLAB interface have been compiled
correctly. Numerical codes obviously usually do not execute identically on
different machines. Therefore small changes in the computations and
consequently in the output do occur. If there really is a difference between
the expected output and the output produce by your compilation you will see
the difference produce by \texttt{diff}. You will most probably see
something like this:

\begin{quote}
\begin{verbatim}
63c63
< 10 4.020936e-10
---
> 10 4.020935e-10
65c65
< 12 2.528134e-12
---
> 12 2.528133e-12
check difference in output!!!
\end{verbatim}
\end{quote}


Anything much worse than this points to an error. Note that the test may
still be considered as "passed" by make.

The same might apply for the MATLAB interface, even for differing versions
of MATLAB:

\begin{quote}
\begin{verbatim}
3,5c3,5
<                   Copyright 1984-2000 The MathWorks, Inc.
<                         Version 6.0.0.88 Release 12
<                                 Sep 21 2000
---
>                   Copyright 1984-2001 The MathWorks, Inc.
>                        Version 6.1.0.450 Release 12.1
>                                 May 18 2001
check difference in output!!!
\end{verbatim}
\end{quote}


% \paragraph{Cleaning the source tree}
% \paragraph{Installing libraries, binaries and documentation}

\subsubsection{Remarks}

The BICGSTAB routine is only intended for testing.  It is not efficient in a
parallel environment because it uses a sparse matrix-vector multiply that
does not scale well.  There is a PETSc interface to SPAI, please consult the
PETSc documentation.

% There will soon be a PETSc interface for version 3 of SPAI.  PETSc is a
% parallel system of linear solver and preconditioners developed, maintained,
% and distributed by Argonne National Laboratory. (See the URL
% http://www-unix.mcs.anl.gov/petsc/.)

BICGSTAB may sometimes produces slightly different results when run on
different numbers of processors.  This is not a bug.  The arithmetic
operations are done in different order, and can therefore lead to different
convergence histories.  SPAI always produces the same preconditioning matrix
on any number of processors when the \texttt{"-bs 1"} option is used.  This
is because the scalar-valued matrix is distributed before blocking is done
and the distribution might break up blocks.

To find good SPAI parameters for your application we suggest that you start
with a fairly large value for $eps$, like 0.7, and then decrease $eps$
until you find the best setting.  Start the -ns parameter as something large,
like 20, and use the -mf parameter to see how many columns don't achieve
$eps$ (if any).  It may be most practical to use a value for -ns that
results in some columns not achieving $eps$.  We also encourage you to
try the different -bs options available.

You can write the matrices (either $A$ or $M$) with a routine called
"\texttt{write\_mm\_matrix}".  Look for some commented-out code in
\texttt{test\_spai.c} (in driver) to see how this is done.

% This code should work on any parallel system using standard MPI.  However,
% it is very sensitive to latency.

Finally, if you are modifying the code there is a debugging trick that is
very helpful.  If you use an option "-db 1" the program will create n files:
dbg0, dbg1, ..., where n is the number of processors.  If you insert code
like the following into your program you can print out information from
different processors in these files.

\begin{quote}
\begin{verbatim}
if (debug) {
  fprintf(fptr_dbg,...);
  fflush(fptr_dbg);
}
\end{verbatim}
\end{quote}
See the file \texttt{spai.c} in \texttt{lib} for an example.


\section{FAQ}

\subsection{Why is SPAI taking forever to compute the preconditioner?}

You are probably using too low a value for $eps$.  You should start with a
relatively large value, say $eps$=0.7 or 0.8, and progressively reduce $eps$
until the total execution time starts to increase again.  This is usually the
value of $eps$ for which the time spent to compute the preconditioner is
approximately equal to that spent in the iterative solver.  Usually the
optimal value for $eps$ lies between 0.5 and 0.7.


\subsection{When I reduce $eps$ the preconditioner does not really improve, why?}

If you reduce $eps$ and the preconditioner does not improve, rerun SPAI with
the \texttt{-mf my\_file} option and check whether any messages have been
written to my\_file. If any columns in the preconditioner did not satisfy
the $eps$ criterion, \texttt{my\_file} will say so, and you need to increase
the number of improvement steps with the -ns option (try 5, 10, 20), before
decreasing $eps$ any further.  However, if \texttt{my\_file} is empty simply
proceed in further reducing $eps$.


\subsection{What does SPAI stand for? }

SPAI stands for SParse Approximate Inverse. The name was invented some time
in the Spring of 1994 during a typical lunch outside on the beautiful lawn
of the Main Quad at Stanford University: an improvised "d\'ejeuner sur
l'herbe" amongst Rodin sculptures below sunny Californian skies. Present
were M. Grote, T. Huckle (TU-Munich), and A.-J. van der Veen (TU-Delft).


% \subsection{On which parallel architecture does SPAI 3.0 yield good performance? }

% You can expect excellent scaling with increasing number of processors on the
% CRAY T3E and the SGI Origin2000, if you are using the SHMEM interface.  You
% can expect good scaling also on the IBM SP2. However you should expect poor
% scaling on a typical network of workstations, because the present version of
% the code is very sensitive to latency (despite caching) due to the large
% number of short messages sent between processors.


\subsection{Why does the preconditioner tend to get worse as I increase the
  block size while keeping $eps$ fixed?}

For the same value of $eps$ an increase in the block size often results in a
slight deterioration in the quality of the preconditioner; this is usually
compensated by a large reduction in computing time. The reason is that the
$eps$ criterion is more stringent in the scalar (bs = 1) than in the block
case. Nevertheless, the overall reduction of $\| AM-I \|_F$ is usually
identical.


\subsection{Does SPAI always work? }

In principle, yes. Unlike many other preconditioners, SPAI cannot break down
if the matrix A is non-singular. Moreover, if one keeps reducing $eps$, SPAI
will eventually compute the exact inverse. Of course this may take a VERY
long time...


\subsection{Configure hangs at \texttt{mexfileextension}-check. Why?}

You probably have the \texttt{DISPLAY} variable not set correctly.
Configure then hangs trying to start MATLAB.  Try starting MATLAB manually.
When successful, try \texttt{configure} again.


\subsection{I still have questions, where do I get further information?}

Please direct your questions to the SPAI mailing list under
\begin{center}
    \url{https://www.maillist.unibas.ch/mailman/listinfo/spai}.
\end{center}

\section{Authors}

% \subsection{Contacts}

% Marcus J. Grote:
% \begin{verbatim}
% Marcus J. Grote
%   Department of Mathematics
%   University of Basel
%   Rheinsprung 21
%   CH-4051 Basel
%   Switzerland
% Tel +41 (0)61 267 39 96
% Fax +41 (0)61 267 3995
% Email: grote@math.unibas.ch
% Homepage: http://www.math.unibas.ch/~grote/
% \end{verbatim}

% \noindent
% Oliver Broeker:
% \begin{verbatim}
% Oliver Broeker
%   Institut fuer Wissenschaftliches Rechnen
%   ETH Zentrum, RZ F9
%   CH-8092 Zuerich
%   Switzerland
% Tel +41 1 632 74 33
% Fax +41 1 632 16 20
% Email: broeker@inf.ethz.ch
% Homepage: http://www.inf.ethz.ch/personal/broeker/
% \end{verbatim}


% \subsection{Used tools}
% This package uses:
% \begin{itemize}
%   \item  BLAS: Basic Linear Algebra subroutines
%   \item  LAPACK: Linear Algebra Package
%   \item  optionally MPI. Message Passing interface
%   \item  GNU autoconf and GNU automake
% \end{itemize}


% \subsection{Thanks}

The main authors of the SPAI code are Stephen Barnard and Marcus Grote.  The
configuration system was programmed by Oliver Br�ker and Michael Hagemann.

We would also like to thank the following people for using SPAI and helping
to improve the package: Victor Eijkhout (PETSc interface), Oliver Ernst, Hans Grote, Jan
Hesthaven, Thomas Huckle, Olaf Schenk, Barry Smith, Bora U\c car.


\addcontentsline{toc}{section}{\protect\numberline{\mbox{}}{References}}


% @PhdThesis{Broeker:2003:PMM,
%   author =	 {Oliver Br�ker},
%   title =	 {Parallel Multigrid Methods using Sparse Approximate
%                   Inverses},
%   school =	 {ETH Zurich},
%   year =	 {2003}
% }

% @Article{Broeker:2002:SAPIN,
%   author =	 {Oliver Br�ker and Marcus~J.~Grote},
%   title =	 {Sparse approximate inverse smoothers for geometric
%                   and algebraic multigrid},
%   journal =	 {Applied Numerical Mathematics},
%   year =	 {2002},
%   volume =	 {41},
%   number =	 {1},
%   pages =	 {61--80}
% }


\begin{thebibliography}{99}
  \bibitem{1} A Block Version of the SPAI Preconditioner, S.T. Barnard and
    M. J.  Grote, in Proc. 9th SIAM Conf. on Parall. Process. for Sci.
    Comp., held March 1999.

  \bibitem{2} A Portable MPI Implementation of the SPAI Preconditioner in
    ISIS++, S.T.  Barnard and R.L. Clay, in Proc. Eighth SIAM Conf. on
    Parallel Process.  for Sci. Comp., held March 1997.

  \bibitem{3} An MPI Implementation of the SPAI Preconditioner on the T3E,
    S. Barnard, L.M. Bernardo and H.D. Simon, technical report LBNL-40794,
    1997.

  \bibitem{4} Parallel Preconditioning with Sparse Approximate Inverses,
    M.J. Grote and T.  Huckle, SIAM J. of Scient. Comput. 18(3), 1997.

  \bibitem{5} Effective Parallel Preconditioning with Sparse Approximate
    Inverses, M.  Grote and T. Huckle, in Proc. Seventh SIAM Conf. on
    Parallel Process.  for Sci. Comp., held February 1995.

  \bibitem{6} Parallel Preconditioning and Approximate Inverses on the
    Connection Machine, M. Grote and H.D. Simon, in Proc. Sixth SIAM Conf.
    on Parallel Process. for Sci. Comp., held March 1993.

  \bibitem{7} Sparse approximate inverse smoothers for geometric and
    algebraic multigrid, O. Br�ker and M.J. Grote, Applied Numerical
    Mathematics 41(1), 2002. %, pp. 61--80.

  \bibitem{8} Parallel Multigrid Methods using Sparse Approximate Inverses,
    O.~Br�ker, PhD Thesis, Dept. of Computer Science, ETH Zurich, 2003.

  \bibitem{9} Robust Parallel Smoothing for Multigrid Via Sparse Approximate
    Inverses, O.~Br�ker, M.~Grote, C.~Mayer, and A.~Reusken, SIAM J. of
    Scient. Comput. 23(4), 2001.

  \bibitem{10} O. Br\"oker and M.J. Grote, Parallel Algebraic
    Multigrid via Sparse Approximate Inverses, in Proc. of 16th IMACS
    World Congress 2000, held August 2000.

\end{thebibliography}

\end{document}
